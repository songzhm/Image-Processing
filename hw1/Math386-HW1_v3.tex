\documentclass[11pt]{article}

%\pagestyle{empty}

\hyphenation{con-si-de-ra-to}

%\textwidth=182mm
%\textheight=230mm
%\hoffset=-31mm                  % horizontal offset
%\voffset=-30mm                  % vertical offset

\usepackage[margin=.775in]{geometry}

\usepackage[usenames]{color}
%\usepackage[mathcal]{eucal}
\usepackage{epsfig}
\usepackage{amsmath}
\usepackage{amssymb}
\usepackage{amsthm}
\usepackage{amsfonts}
\usepackage{enumerate}
\usepackage{amsthm}
\usepackage{dsfont}
\usepackage[usenames]{color}
\usepackage{wrapfig}
\usepackage{graphicx}
\usepackage{blindtext}
\usepackage{enumerate}
\usepackage{fancyvrb}


\usepackage{lscape}

\usepackage{color,hyperref}
\definecolor{darkblue}{rgb}{0.0,0.0,.45}
\definecolor{darkred}{rgb}{0.7,0,0}
\hypersetup{colorlinks,breaklinks,
	linkcolor=darkred,urlcolor=darkred,
	anchorcolor=darkred}


\newcommand{\myparallel}{{\mkern3mu\vphantom{\perp}\vrule depth 0pt\mkern2mu\vrule depth 0pt\mkern3mu}}

\newcommand{\qq}{\mathbf{q}}
\newcommand{\qqd}{\mathbf{\dot{q}}}
\newcommand{\ind}{\mathds{1}} % so $\ind{1}$ gives you the script 1
\newcommand{\gul}[2]{g^{#1 ,}_{\textcolor{White}{#1}#2\ }}

\theoremstyle{plain}
\newtheorem{corollary}{Corollary}
\newtheorem{proposition}{Proposition}
\newtheorem{lemma}{Lemma}
\theoremstyle{definition}
\newtheorem{definition}{Definition}
\newtheorem{example}{Example}
\newtheorem{case}{Case}
\newtheorem{conj}{Conjecture}
\newtheorem*{remark}{Remark}
\newtheorem*{notation}{Notation}
%\newtheorem{fc1}{From Chapter 1 of Millman \& Parker:}
%\newtheorem{fc2}{From Chapter 1 of Millman \& Parker:}
\newtheorem{problem}{Problem}
\newtheorem{theorem}{Theorem}
%\renewcommand*{\theproblem}{\hwn\Alph{problem}}
\renewcommand*{\theproblem}{\hwn.\arabic{problem}}
%\newtheorem{theorem}{Theorem}
\theoremstyle{remark}

\parindent=0cm

\newcommand{\hwn}{1}

\begin{document}
\LARGE
\noindent{\textsf{\textbf{CGU Math 381 (Image Processing), Spring '18 
			-- Homework \#\hwn}}}
\par\vspace{.25cm}
\Large
\hspace{.5cm}\textsf{\mbox{%
Point operations, filters, and partial differential operators. 
}}
\par\vspace{.2cm}
\normalsize
\hspace{.5cm}\mbox{\textsf{\em Due 
		on Tuesday 2/6, at the beginning of class. }%
}
\par\vspace{.3cm}
\normalsize\rm
%$\boxed{\mbox{ \bf This homework assignment is on two
%		 pages. }}$ \par\vspace{.3cm}
\bf Reading: \rm
While the class notes should be sufficient 
for this homework assignment, it is still useful to read 
Chapters~1 to~6 of \em Principles of Digital Image
Processing \em by Burger \& Burge. You may skip
all the parts that refer to Java implementations
and ImageJ (for example, all of Chapter~2), unless they are of interest to you. The most important chapters are, for our purposes, Chapter~3 to
Chapter~6.
\par\vspace{.15cm}
Write, on top of the first page of your assignment:
Name (\underline{LAST}, First), your University or College, the HW\#, and acknowledge other students
with whom you may have worked (just write \mbox{``\em Worked with \ldots\/\em''}). For the computational problems 
where you are asked to write computer code you may choose 
the programming language that you prefer, such as \em Python \em 
(I am learning it myself!) or
\em Matlab\em\/.
%\par\vspace{.15cm}
%Solve the following problems:
\begin{problem}
Suppose that an image has all its pixel values clustered in a small
interval---say, in $[a,b]\subset[0,L-1]$, with $0<a<b<L-1$.
The image needs to be enhanced, so that the small range~$[a,b]$
maps to a the range~$[0 ,L-1]$. What operation 
$S:[0,L-1]\rightarrow[0 ,L-1]$ would you
use for this purpose?
\end{problem}
\begin{problem}[$\gamma$-correction]
Remember that \em gamma correction \em is
a point operation defined as follows:
$$
J(i,j) = (L-1)\Big(\frac{I(i,j)}{L-1}\Big)^\gamma,
\qquad
\mbox{for }
(i,j)\in\mathbb{D}=
\{0,1,\ldots,M\}
\times
\{0,1,\ldots,N\}, 
$$
where $L-1=2^b-1$ is typically 255, and $\gamma $ is a positive 
constant chosen by the user. 
{\bf(a)}~Using software of your choice, plot on the same figure
the graphs of the function
$S_\gamma:[0,L-1]\rightarrow[0,L-1]$, 
where~$s=S_\gamma(r)=(L-1)[r/(L-1)]^{\gamma}$,
for the following values of~$\gamma$:
0.04, 0.1, 0.2, 0.4, 0.61, 1, 1.5, 2.5, 5, 10, and 25.  
{\bf(b)}~Fix an arbitrary value of~$\gamma$: what is the \em inverse \em transformation~$S_\gamma^{-1}$, i.e.~the function such that
$S_\gamma^{-1}(S(r))=r$? 
{\bf(c)} Consider now the images \tt image-spine.tif \rm and 
\tt image-airport.tif\rm.
The first looks too dark, while the other looks too bright. 
Write computer code\footnote{For implementation purposes, 
note that we can write $a^{\gamma}=e^{\gamma\ln a}$.} 
to adjust the brightness of the images by applying 
gamma corrections with appropriate values of the 
parameter $\gamma$ (remember that the output values must be 
rounded off so to be in the set $\mathbb{P}=\{0,1,2,\ldots,L-1\}$). 
Show your results, and also plot the \em histograms \em of
the original images as well as the transformed images. 
Provide your code.
\end{problem}
\begin{problem}
Apply a $3\times3$ averaging filter to the image
\tt image-lena-noisy.png\rm, which is the classic
Lena\footnote{\em Lena \em is the name given to a standard test image widely used in the field of image processing since 1973. It is in fact a photo of Lena S\"oderberg, shot by photographer Dwight Hooker, cropped from the centerfold of the November 1972 issue of Playboy magazine (!). 
She was a guest at the \mbox{50th annual Conference of the Society for Imaging Science and Technology in 1997.}} image with 
added Gaussian noise. Also try a $5\times5$ uniform filter, and a non-uniform 
averaging $5\times5$ filter (`Gaussian mask').
\mbox{In general, what do you observe in your result if
you apply larger filter masks?}  
\end{problem}
\begin{problem}
The median filter (see the class notes) is the most suitable 
to deal with \em salt-and-pepper \em noise. 
{\bf(a)}~Show that a median filter is \em nonlinear\em\/.
{\bf(b)}~Write computer code to apply a $3\times3$ median filter to \tt image-circuit.jpg\rm, 
which is an x-ray image of a circuit board corrupted by salt-and-pepper
noise. 
{\bf(c)}~Now apply a $3\times3$ linear average filter to \tt image-circuit.jpg\rm. Why is the result so different?
\end{problem}
\begin{problem}
	\label{assoc}
Show that convolution is an \em associative \em operation,
i.e.~that for three arbitrary functions $f,g,h:\mathbb{Z}^2\rightarrow\mathbb{R}$ it is the case that 
$f*(g*h)=(f*g)*h$.
\end{problem}
\begin{problem}[3-point midpoint formula for the second derivative]
\label{sder}
Remember from class (and Numerical Analysis, if you have taken that course) the so-called \em forward difference \em 
and \em backward difference \em formulas: 
\par\vspace{.1cm}
\hfill
$
\displaystyle
f'(x)\simeq\frac{f(x+\Delta x)-f(x)}{\Delta x}
\qquad
\mbox{and}
\qquad
f'(x)\simeq\frac{f(x)-f(x-\Delta x)}{\Delta x}
$
\hfill\mbox{}\par\vspace{.1cm}
for numerical differentiation. 
Combine the two to find the 3-point midpoint formula
for the numerical computation of the second derivative
of a function,
namely
$f''(x)\simeq(f(x+\Delta x)-2f(x)+f(x-\Delta x))/(\Delta x)^2$.
\end{problem}
\begin{problem}
	The result of Problem~\ref{assoc} 
	shows that if we want to apply two filters
	(with convolution kernels $g$ and $h$, respectively)
	to a given image~$I$ one after the other, it is the case
	that $(I\ast g)\ast h = I\ast(g\ast h)$.
	In other words, we can combine the two convolution
	kernels into one, $f=g\ast h$, and then filter
	the image with the \em new \em kernel by computing~$I\ast f$. 
	In class we introduced several kernels for computing
	$\frac{\partial I}{\partial x}$, namely: 
	$$
	h_x = 
	\left[
	\begin{array}{ccc}
	h_x(-1,-1) & h_x(-1,0) & h_x(-1,1) \\
	h_x(0,-1) & h_x(0,0) & h_x(0,1) \\
	h_x(1,-1) & h_x(1,0) & h_x(1,1) 
	\end{array}
	\right]
	=
	\frac{1}{2}
	\left[
	\begin{array}{ccc}
	0 & 1 & 0 \\
	0 & 0 & 0 \\
	0 & -1 & 0 
	\end{array}
	\right],
	$$
	$$
	h^P_x = 
	\frac{1}{6}
	\left[
	\begin{array}{ccc}
	1 & 1 & 1 \\
	0 & 0 & 0 \\
	-1 & -1 & -1 
	\end{array}
	\right],
 	\qquad
 	h^S_x = 
 	\frac{1}{8}
 	\left[
 	\begin{array}{ccc}
 	1 & 2 & 1 \\
 	0 & 0 & 0 \\
 	-1 & -2 & -1 
 	\end{array}
 	\right].
	$$
The first one is derived from the 3-point midpoint formula
for numerical differentiation,
the second one is the \em Prewitt \em filter
(which is obtained from $h_x$ by averaging over 3
horizontal pixels, with equal weights),
while the last one is the \em Sobel \em filter
(which is obtained from $h_x$ by averaging over 3
horizontal pixels, with weights 1/4, 1/2, and 1/4).
We also introduced the following filter
for the \em second \em partial derivative
$\frac{\partial^2}{\partial x^2}$:
$$
h_{xx} 
= 
\left[
\begin{array}{ccc}
h_{xx}(-1,-1) & h_{xx}(-1,0) & h_{xx}(-1,1) \\
h_{xx}(0,-1) & h_{xx}(0,0) & h_{xx}(0,1) \\
h_{xx}(1,-1) & h_{xx}(1,0) & h_{xx}(1,1) 
\end{array}
\right]
= 
\left[
\begin{array}{ccc}
0 & 1 & 0 \\
0 & -2 & 0 \\
0 & 1 & 0 
\end{array}
\right],
$$
for which we used the 3-point midpoint formula for the second derivative,
which you found in Problem~\ref{sder}. However,
since we can compute a second partial derivative by performing
a single partial derivative \em twice \em
in the same direction, the above reasoning suggests that we may 
approximate $\partial^2I/\partial x^2$ with
$I\ast f$: here, $f=h\ast g$ where $h$ and $g$
may be chosen independently as either~$h_x$, $h^P_x$, or
$h^S_x$. In how many ways can this be done?
Also, what are the effects of choosing $f=h_x\ast h_x$
rather than, say, $f=h_x\ast h^P_x$? Compute, either by
hand or using some computer code, the convolution kernel
$f=h_x\ast h_x^S$. What is its size?
\end{problem}
\begin{problem}[Mixed partials]
	Suggest a convolution kernel~$h_{xy}$
	such that 
	$
	\displaystyle 
	\frac{\partial^2 I}{\partial x\partial y}
	\simeq I\ast h_{xy}$.
	\par\vspace{-.15cm}
	{\em Hint:}
	There are multiple correct answers to this question.
\end{problem}
\begin{problem}(Rotation invariance of the Laplacian operator.)
Show that the Laplacian operation 
$\nabla^2f=\frac{\partial^2f}{\partial x^2}+\frac{\partial^2f}{\partial y^2}$ is \em isotropic \em (i.e.~invariant under rotations, or rotationally invariant). That is, we choose an arbitrary 
angle~$\theta$ and operate the change of 
coordinates $T:(x,y)\mapsto(u,v)$ given by the transformation: 
$$
T:
\left\{
\begin{array}{ccl}
x(u,v) & = & u\cos\theta - v \sin\theta \\
y(u,v) & = & u\sin\theta + v \cos\theta 
\end{array}
\right.
\qquad
\Longleftrightarrow
\qquad
T^{-1}:
\left\{
\begin{array}{ccl}
u(x,y) & = & x\cos\theta + y \sin\theta \\
v(x,y) & = & -x\sin\theta + y \cos\theta 
\end{array}
\right.
$$ 
where $(x,y)$ are the unrotated and $(u,v)$ are the rotated coordinates.
Define the composite function $F(u,v)=f(x(u,v),y(u,v))$, 
which is nothing but $f$ expressed in the new coodrinates, and show that
$
\frac{\partial^2 F}{\partial u^2}(u,v)+\frac{\partial^2 F}{\partial v^2}(u,v)
=
\frac{\partial^2 f}{\partial x^2}(x(u,v),y(u,v))+\frac{\partial^2 f}{\partial y^2}(x(u,v),y(u,v))
$. Equivalently, since we have $f(x,y)=F(u(x,y),v(x,y))$, you can show that
$
\frac{\partial^2 f}{\partial x^2}(x,y)+\frac{\partial^2 f}{\partial v^2}(x,y)
=
\frac{\partial^2 F}{\partial u^2}(u(x,y),v(x,y))+\frac{\partial^2 F}{\partial v^2}(u(x,y),v(x,y))
$). That is, the Laplacian operator ``looks the same''
in a new coordinate system that is obtained by rotating the original one
by an arbitrary angle~$\theta$. This is what we call \em isotropy\em\/, or \em rotational invariance \em of $\nabla^2$.
\end{problem}
\begin{problem}[A basic sharpening filter.]
Write a computer program that implements the operation 
$J(x,y)=I-\nabla^2 I$, in the form $J=I\ast h^{\mathrm{Sh}}$ with
a $3\times 3$ sharpening convolution kernel~$h^{\mathrm{Sh}}$. Give the form of the convolution kernel
apply the program to the image of the
North Pole of the moon (\tt image-moon.jpg\rm). You should turn in the the computer code, the input and output images, as well
as a short explanation of why the method works. 
\end{problem}
\end{document}













%%% Local Variables:
%%% mode: latex
%%% TeX-master: t
%%% End:
